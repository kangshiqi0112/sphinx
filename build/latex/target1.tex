%% Generated by Sphinx.
\def\sphinxdocclass{article}
\documentclass[letterpaper,10pt,english]{sphinxhowto}
\ifdefined\pdfpxdimen
   \let\sphinxpxdimen\pdfpxdimen\else\newdimen\sphinxpxdimen
\fi \sphinxpxdimen=.75bp\relax
%% turn off hyperref patch of \index as sphinx.xdy xindy module takes care of
%% suitable \hyperpage mark-up, working around hyperref-xindy incompatibility
\PassOptionsToPackage{hyperindex=false}{hyperref}

\PassOptionsToPackage{warn}{textcomp}

\catcode`^^^^00a0\active\protected\def^^^^00a0{\leavevmode\nobreak\ }
\usepackage{cmap}
\usepackage[LGR,X2,T1]{fontenc}
\usepackage{amsmath,amssymb,amstext}
\usepackage{babel}



\usepackage[sc]{mathpazo}
\usepackage[scaled]{helvet}
\usepackage{courier}


\usepackage[Sonny]{fncychap}
\ChNameVar{\Large\normalfont\sffamily}
\ChTitleVar{\Large\normalfont\sffamily}
\usepackage{sphinx}

\fvset{fontsize=\small}
\usepackage{geometry}


% Include hyperref last.
\usepackage{hyperref}
% Fix anchor placement for figures with captions.
\usepackage{hypcap}% it must be loaded after hyperref.
% Set up styles of URL: it should be placed after hyperref.
\urlstyle{same}

\addto\captionsenglish{\renewcommand{\contentsname}{Contents:}}

\usepackage{sphinxmessages}
\setcounter{tocdepth}{2}


\hypersetup{unicode=true}
\usepackage{ctex}


\title{title \unexpanded{说明}}
\date{2020 年 12 月 15 日}
\release{0.1}
\author{kk}
\newcommand{\sphinxlogo}{\vbox{}}
\renewcommand{\releasename}{发布}
\makeindex
\begin{document}

\ifdefined\shorthandoff
  \ifnum\catcode`\=\string=\active\shorthandoff{=}\fi
  \ifnum\catcode`\"=\active\shorthandoff{"}\fi
\fi

\pagestyle{empty}
\sphinxmaketitle
\pagestyle{plain}
\sphinxtableofcontents
\pagestyle{normal}
\phantomsection\label{\detokenize{index::doc}}



\section{welcome}
\label{\detokenize{welcome:welcome}}\label{\detokenize{welcome::doc}}
通过使用标点符号为节标题(ref)加下划线(上划线可选)来创建节标题, 至少与文本一样长:
sphinx 使用restructuredText作为他的标记语言
\begin{itemize}
\item {} 
\sphinxstylestrong{输出格式:} 超文本标记语言

\item {} 
\sphinxstylestrong{输出格式:} 超文本标记语言

\item {} 
\sphinxstylestrong{一个*:} \sphinxstyleemphasis{表示斜体}

\item {} 
\sphinxstylestrong{两个‘}’:** 用于强调(粗体)

\item {} 
\sphinxstylestrong{代码两个反引号:}\sphinxcode{\sphinxupquote{include dddddd}}

\end{itemize}
\begin{itemize}
\item {} 
this is

\item {} 
a list
\begin{itemize}
\item {} 
with a nested list

\item {} 
and some subitems

\end{itemize}

\item {} 
and here the parent list continues

\item {} 
嵌套列表,但注意它们必须通过空行与父列表项分开:

\end{itemize}

This is a normal text paragraph. The next paragraph is a code sample:

\begin{sphinxVerbatim}[commandchars=\\\{\}]
\PYG{n}{It} \PYG{o+ow}{is} \PYG{o+ow}{not} \PYG{n}{processed} \PYG{o+ow}{in} \PYG{n+nb}{any} \PYG{n}{way}\PYG{p}{,} \PYG{k}{except}
\PYG{n}{that} \PYG{n}{the} \PYG{n}{indentation} \PYG{o+ow}{is} \PYG{n}{removed}\PYG{o}{.}

\PYG{n}{It} \PYG{n}{can} \PYG{n}{span} \PYG{n}{multiple} \PYG{n}{lines}\PYG{o}{.}
\end{sphinxVerbatim}

This is a normal text paragraph again.


\begin{savenotes}\sphinxattablestart
\centering
\begin{tabulary}{\linewidth}[t]{|T|T|T|}
\hline
\sphinxstyletheadfamily 
A
&\sphinxstyletheadfamily 
B
&\sphinxstyletheadfamily 
A and B
\\
\hline
False
&
False
&
False
\\
\hline
True
&
False
&
False
\\
\hline
False
&
True
&
False
\\
\hline
True
&
True
&
True
\\
\hline
\end{tabulary}
\par
\sphinxattableend\end{savenotes}

This is a paragraph that contains \sphinxhref{https://domain.invalid/}{a link}%
\begin{footnote}[1]\sphinxAtStartFootnote
\sphinxnolinkurl{https://domain.invalid/}
%
\end{footnote}.

\noindent\sphinxincludegraphics{{1}.png}

对于脚注(ref),使用 {[}\#name{]}\_ 标记脚注位置,并在 “Footnotes ” 标题后添加脚注主体在文档底部,像这样:

Lorem ipsum %
\begin{footnote}[2]\sphinxAtStartFootnote
Text of the first footnote.
%
\end{footnote} dolor sit amet … %
\begin{footnote}[3]\sphinxAtStartFootnote
Text of the second footnote.
%
\end{footnote}

支持标准reST引用(ref),其附加功能是 “global” ,即所有引用都可以从所有文件引用。像这样使用它们:

Lorem ipsum \sphinxcite{welcome:ref} dolor sit amet.


\section{说明}
\label{\detokenize{_u8bf4_u660e:id1}}\label{\detokenize{_u8bf4_u660e::doc}}
\sphinxstylestrong{使用说明}
\begin{itemize}
\item {} 
在index.rst文件中引入新增加的文件

\item {} 
执行命令:sphinx\sphinxhyphen{}build \sphinxhyphen{}b html source build

\item {} 
执行命令:make html

\item {} 
pdf生成:make latexpdf

\end{itemize}

\sphinxstylestrong{介绍}
\begin{itemize}
\item {} 
source文件夹:存放源文件(rst/txt/md)、资源文件(图片等)、配置文件(conf.py)

\item {} 
build文件夹:编译后的文件(HTML、PDF)

\end{itemize}

\sphinxstylestrong{相关链接}
\begin{itemize}
\item {} 
主题:\sphinxurl{http://www.pythondoc.com/sphinx/theming.html\#using-a-theme}

\item {} 
主题: \sphinxhref{http://www.pythondoc.com/sphinx/theming.html\#using-a-theme/}{a link}%
\begin{footnote}[4]\sphinxAtStartFootnote
\sphinxnolinkurl{http://www.pythondoc.com/sphinx/theming.html\#using-a-theme/}
%
\end{footnote}

\item {} 
texlive镜像资源:\sphinxhref{https://mirrors.tuna.tsinghua.edu.cn/CTAN/systems/texlive/Images/}{link}%
\begin{footnote}[5]\sphinxAtStartFootnote
\sphinxnolinkurl{https://mirrors.tuna.tsinghua.edu.cn/CTAN/systems/texlive/Images/}
%
\end{footnote}

\item {} 
配置markdown:\sphinxhref{https://www.sphinx-doc.org/en/master/usage/markdown.html}{markdown}%
\begin{footnote}[6]\sphinxAtStartFootnote
\sphinxnolinkurl{https://www.sphinx-doc.org/en/master/usage/markdown.html}
%
\end{footnote}

\item {} 
\sphinxurl{https://zhuanlan.zhihu.com/p/136931926}

\item {} 
texlive: \sphinxurl{https://blog.csdn.net/weixin\_39892850/article/details/105468247}

\end{itemize}


\section{title.txt}
\label{\detokenize{12.8:title-txt}}\label{\detokenize{12.8::doc}}
安装:
Python –version
pip install \sphinxhyphen{}U sphinx
sphinx\sphinxhyphen{}build –version

产生文件夹源目录:
sphinx\sphinxhyphen{}quickstart
(https://www.sphinx\sphinxhyphen{}doc.org/en/master/usage/configuration.html\#confval\sphinxhyphen{}language.)
(maxdepth: 2—文档深度)

运行构建:
\$ sphinx\sphinxhyphen{}build \sphinxhyphen{}b html sourcedir builddir
(sphinx\sphinxhyphen{}build.exe \sphinxhyphen{}b html source build)
./make.bat html(如果需要发布)

sphinx\sphinxhyphen{}build \sphinxhyphen{}b html source build
make html

安装支持markdown
pip install –upgrade recommonmark


\section{one file.md}
\label{\detokenize{aa:one-file-md}}\label{\detokenize{aa::doc}}

\subsection{二级标题}
\label{\detokenize{aa:id1}}

\subsubsection{三级标题}
\label{\detokenize{aa:id2}}
\sphinxstylestrong{这是加粗的文字}
\sphinxstyleemphasis{这是倾斜的文字}`
\sphinxstyleemphasis{\sphinxstylestrong{这是斜体加粗的文字}}


\section{Indices and tables}
\label{\detokenize{index:indices-and-tables}}\begin{itemize}
\item {} 
\DUrole{xref,std,std-ref}{genindex}

\item {} 
\DUrole{xref,std,std-ref}{modindex}

\item {} 
\DUrole{xref,std,std-ref}{search}

\end{itemize}

\begin{sphinxthebibliography}{Ref}
\bibitem[Ref]{welcome:ref}
Book or article reference, URL or whatever.
\end{sphinxthebibliography}



\renewcommand{\indexname}{索引}

\IfFileExists{\jobname.ind}
             {\footnotesize\raggedright\printindex}
             {\begin{sphinxtheindex}\end{sphinxtheindex}}

\end{document}